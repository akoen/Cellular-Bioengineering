\documentclass{article} % A4 paper and 11pt font size
\usepackage[T1]{fontenc} % Use 8-bit encoding that has 256 glyphs
\usepackage{mathpazo}

\usepackage[letterpaper, margin=1.25in]{geometry}
\usepackage{parskip}
\usepackage{setspace}
\usepackage{titlesec}
\titleformat{\section}[hang]{\normalfont\scshape\bfseries}{\thesection}{1em}{}
\titleformat{\subsection}[runin]{\normalfont\scshape}{\thesection}{1em}{}
\titlespacing{\section}{0pt}{15pt}{1em}[]

\usepackage{hyperref}
\usepackage[document]{ragged2e}

\usepackage{siunitx}
\usepackage{amsmath}

\usepackage{fancyhdr} % Custom headers and footers
\pagestyle{fancyplain} % Makes all pages in the document conform to the custom headers and footers
\fancyhead{} % No page header - if you want one, create it in the same way as the footers below
\fancyfoot[L]{} % Empty left footer
\fancyfoot[C]{} % Empty center footer
\fancyfoot[R]{\thepage} % Page numbering for right footer
\renewcommand{\headrulewidth}{0pt} % Remove header underlines
\renewcommand{\footrulewidth}{0pt} % Remove footer underlines
\setlength{\headheight}{13.6pt} % Customize the height of the header

\setlength\parindent{0pt} % Removes all indentation from paragraphs - comment this line for an assignment with lots of text

% ----------------------------------------------------------------------------------------
%	TITLE SECTION
% ----------------------------------------------------------------------------------------

\newcommand{\horrule}[1]{\rule{\linewidth}{#1}} % Create horizontal rule command with 1 argument of height

\title{
  \normalfont \normalsize
  \textsc{The University of British Columbia} \\ [25pt] % Your university, school and/or department name(s)
  \horrule{0.5pt} \\[0.4cm] % Thin top horizontal rule
  \huge BMEG 101 - Cellular and Molecular Bioengineering Project Assignment % The assignment title
  \horrule{2pt} \\[0.5cm] % Thick bottom horizontal rule
}

\author{Alex Koen \& Chloe Bolongaro} % Your name

\date{\normalsize \today} % Today's date or a custom date

\begin{document}

\maketitle % Print the title

\onehalfspacing

\begin{enumerate}
    \item 
    \item 
    \item \textit{Calculate the decay length, $\lambda$ of the BMP4 concentration gradient}
    
    \begin{align}
        \lambda &= \sqrt{\frac{D}{k}} \\
        &= \sqrt{\frac{\SI{1.2}{\micro\m^2\per\s}}{\SI{5.4e-5}{\per\s}}} \\
        &= \SI{149}{\micro\m} \\
    \end{align}
    \items
    \item 
    \item
    \item 
    \item To increase the size of the BRA+ region, one could increase the temperature.
    
    The approximate change in the diffusion coefficient as a result of an increase in temperature is given by the Stokes-Einstein equation:
    
    \begin{align}
        \frac{D_{T_1}}{D_{T_2}}=\frac{T_1}{T_2} \frac{\mu_{T_1}}{\mu_{T_2}}
    \end{align}
    \item 
\end{enumerate}


\end{document}

\implies
%%% Local Variables:
%%% mode: latex
%%% TeX-engine: xetex
%%% End:
