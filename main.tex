\documentclass{article} % A4 paper and 11pt font size
\usepackage[T1]{fontenc} % Use 8-bit encoding that has 256 glyphs
\usepackage{mathpazo}

\usepackage[letterpaper, margin=1.25in]{geometry}
\usepackage{parskip}
\usepackage{setspace}
\usepackage{titlesec}
\titleformat{\section}[hang]{\normalfont\scshape\bfseries}{\thesection}{1em}{}
\titleformat{\subsection}[runin]{\normalfont\scshape}{\thesection}{1em}{}
\titlespacing{\section}{0pt}{15pt}{1em}[]

\usepackage[document]{ragged2e}

\usepackage{siunitx}
\usepackage{amsmath}

\usepackage[style=apa,sortcites=true,sorting=nyt,backend=biber]{biblatex}
\addbibresource{ref.bib}

\usepackage{fancyhdr} % Custom headers and footers
\pagestyle{fancyplain} % Makes all pages in the document conform to the custom headers and footers
\fancyhead{} % No page header - if you want one, create it in the same way as the footers below
\fancyfoot[L]{} % Empty left footer
\fancyfoot[C]{} % Empty center footer
\fancyfoot[R]{\thepage} % Page numbering for right footer
\renewcommand{\headrulewidth}{0pt} % Remove header underlines
\renewcommand{\footrulewidth}{0pt} % Remove footer underlines
\setlength{\headheight}{13.6pt} % Customize the height of the header

\setlength\parindent{0pt} % Removes all indentation from paragraphs - comment this line for an assignment with lots of text

% ----------------------------------------------------------------------------------------
%	TITLE SECTION
% ----------------------------------------------------------------------------------------

\newcommand{\horrule}[1]{\rule{\linewidth}{#1}} % Create horizontal rule command with 1 argument of height

\title{
  \normalfont \normalsize
  \textsc{The University of British Columbia} \\ [25pt] % Your university, school and/or department name(s)
  \horrule{0.5pt} \\[0.4cm] % Thin top horizontal rule
  \huge BMEG 101 - Cellular and Molecular Bioengineering Project Assignment % The assignment title
  \horrule{2pt} \\[0.5cm] % Thick bottom horizontal rule
}

\author{Alex Koen \& Chloe Bolongaro} % Your name

\date{\normalsize \today} % Today's date or a custom date

\begin{document}

\maketitle % Print the title

\onehalfspacing

\begin{enumerate}
\item \textit{Design of the 1-dimensional cell culture system. Explain appropriate size and materials you will use, and provide a diagram of the set up.}

  Stem cell cultures are delicate, and depend on external support for cell survival and to allow for self-renewal. The 1-dimensional cell culture used in this project will be composed of a polycaprolactone outer structure, which has been shown to act effectively as a bioinert scaffold \parencite{bertucci2018biomaterial}. While it would be ideal to coat the scaffold with a full extracellular matrix, this can be delicate. Instead, the inner wall of the tube will be coated with fibronectin to support cell attachment.
\item 
\item \textit{Calculate the decay length, $\lambda$ of the BMP4 concentration gradient}
  
  \begin{align*}
    \lambda &= \sqrt{\frac{D}{k}} \\
            &= \sqrt{\frac{\SI{1.2}{\micro\m^2\per\s}}{\SI{5.4e-5}{\per\s}}} \\
            &= \SI{149}{\micro\m} \\
  \end{align*}
\item Write differential equations for the dynamics of the BMP4 concentration and BMPR2 concentration.

  \textbf{Dynamics of BMP4:}
  
  \begin{align*}
    L &= L_0 e^{\frac{-x}{\lambda}} \\
    \intertext{This shows that the ligand concentration decreases exponentially along the tube. Now, taking the second derivative of $L$ with respect to $x$:}
    \frac{\partial L}{\partial x} &= \frac{-L_0}{\lambda}e^{\frac{-x}{\lambda}} \\
    \frac{\partial ^2L}{\partial x^2} &= \frac{L_0}{\lambda^2}e^{\frac{-x}{\lambda}}
                                        \intertext{Given $\frac{\partial L }{\partial t} = D\frac{\partial^2 L}{\partial x^2}-kL+\nu$ (Fick's second law with a sink and a source):}
                                        \frac{\partial L}{\partial t} &= \frac{DL_0}{\lambda^2}e^{\frac{-x}{\lambda}}-kL_0e^{\frac{-x}{\lambda}} + \nu \\
      &= L_0e^{\frac{-x}{\lambda}}\left(\frac{D}{\lambda^2} -k \right) + \nu
  \end{align*}
  
  \textbf{Dynamics of BMPR2:}

  It is know that $R$ is produced linearly proportional to [BMP4] with proportionality constant $\alpha$ and degraded with rate constant $K_R$.

  \begin{align*}
    \frac{dR}{dt} &= \alpha L-k_R R \\
    \intertext{and given}
    L &= L_0 e^{\frac{-x}{\lambda}} \\
    \frac{dR}{dt} &= \alpha L_0 e^{\frac{-x}{\lambda}} -k_R R \\
  \end{align*}

\item \textit{Find solutions for ligand and receptor equations in steady state.}

  $L_0=\SI{2}{\nano\g\per\micro\m}$

  $\alpha = \SI{5e-3}{\per\s}$

  $k_R= \SI{10e-3}{\per\s}$


  \textbf{Solution for ligand equation in steady-state:}

  In steady state, $\frac{\partial L}{\partial t} = 0$  

  \begin{align*}
    0 &= L_0e^{\frac{-x}{\lambda}}\left(\frac{D}{\lambda^2} -k \right) + \nu \\
      &= L_0e^{\frac{-x}{\lambda}}\left(k - \frac{D}{\lambda^2} \right) + \nu \\
    \nu &= \SI{2}{\nano\g\per\micro\m}e^{\frac{-x}{149}} \left (\SI{5.4e-5}{\per\s} - \frac{\SI{1.2}{\micro\m^2\per\s}}{(\SI{149}{\micro\m})^2} \right ) \\
    \nu &= 0 \\
    \intertext{Additionally, since $k - \frac{D}{\lambda^2}=0$ it can be inferred that:}
    \frac{\partial L}{\partial t} &= \nu \\
    \intertext{and}
    \int \frac{\partial L}{\partial t}~dt&=\int \nu~dt \\ 
    \Delta L &= \left [\frac{\nu^2}{2}\right ]_{t_1}^{t_2}
               \intertext{If the source is held constant:}
               \Delta L &= \nu t
  \end{align*}

  This makes sense---the only way for the ligand concentration to be constant over time is if there \emph{is no} source of ligands. Also note that the rate of ligand diffusion is \emph{equal} to its effective degradation rate, given by $kL$. Consequently, the change in ligand concentration over time is independent of $x$ and determined solely by the source.
  

  \textbf{Solution for receptor equation in steady state:}

  \begin{align*}
    \frac{dR}{dt} &= \alpha L_0 e^{\frac{-x}{\lambda}} -k_R R \\
    0 &= \alpha L_0 e^{\frac{-x}{\lambda}} -k_R R \\
    R &= \frac{\alpha}{k_R} L_0e^{\frac{-x}{\lambda}} \\
    \intertext{at the end of the tube ($x=0$):}
    R &= \frac{\alpha}{k_R}L_0 \\
                  &= \frac{\SI{5e-3}{\per\s}}{\SI{10e-3}{\per\s}} \SI{2}{\nano\gram\per\micro\m} \\
                  &= \SI{1}{\nano\g\per\micro\m}
  \end{align*}
  
\item \textit{Considering the association constant $K_a$, calculate the concentration of the ligand-receptor complex in steady state at the distance from the source $x=\lambda$:}

  \begin{align*}
    K_a &= \frac{K_{on}}{K_{off}} \\
    [RL] &= K_a[R][L] \\
    \intertext{$L$ at $x=\lambda$:}
    L &= \frac{L_0}{e} \\ 
        &= \frac{\SI{2}{\nano\g\per\micro\m}}{e} \\
        &= \SI{0.74}{\nano\g\per\micro\m} \\
    \intertext{$R$ at $x=\lambda$:}
    R &= \frac{\alpha L}{k_R} \\
        &= \frac{\SI{5e-3}{\per\s}\cdot \SI{0.74}{\nano\g\per\micro\m}}{\SI{10e-3}{\per\s}} \quad \text{(from above)} \\
        &= \SI{0.37}{\nano\g\per\micro\m}
          \intertext{It follows that:}
          [RL] &= K_a\cdot\SI{0.74}{\nano\g\per\micro\m}\cdot\SI{0.37}{\nano\g\per\micro\m} \\
        &= K_a \cdot\SI{0.27}{\nano\g^2\per\micro\m^2}
  \end{align*}
\item 
\item \textit{Discuss how you would modify the size of the BRA+ region.}


  The size of the BRA+ region is given by the decay length $\lambda$. To increase the size of the BRA+ region, one must increase lambda.
  
\item \textit{Discuss how therapeutics can benefit from a precise control of differentiation patterns.}

  Therapeutics can benefit from precise control of differentiation patterns because such control would allow them to genetically engineer tissues and organs with specific cell types in a predetermined configuration.
\end{enumerate}

% \printbibliography

\end{document}

\implies
%%% Local Variables:
%%% mode: latex
%%% TeX-engine: xetex
%%% End:
