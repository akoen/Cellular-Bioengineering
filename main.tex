\documentclass{article} % A4 paper and 11pt font size
\usepackage[T1]{fontenc} % Use 8-bit encoding that has 256 glyphs
\usepackage{mathpazo}

\usepackage[letterpaper, margin=1.25in]{geometry}
\usepackage{parskip}
\usepackage{setspace}
\usepackage{titlesec}
\titleformat{\section}[hang]{\normalfont\scshape\bfseries}{\thesection}{1em}{}
\titleformat{\subsection}[runin]{\normalfont\scshape}{\thesection}{1em}{}
\titlespacing{\section}{0pt}{15pt}{1em}[]

\usepackage{hyperref}
\usepackage[document]{ragged2e}

\usepackage{siunitx}
\usepackage{amsmath}

\usepackage{fancyhdr} % Custom headers and footers
\pagestyle{fancyplain} % Makes all pages in the document conform to the custom headers and footers
\fancyhead{} % No page header - if you want one, create it in the same way as the footers below
\fancyfoot[L]{} % Empty left footer
\fancyfoot[C]{} % Empty center footer
\fancyfoot[R]{\thepage} % Page numbering for right footer
\renewcommand{\headrulewidth}{0pt} % Remove header underlines
\renewcommand{\footrulewidth}{0pt} % Remove footer underlines
\setlength{\headheight}{13.6pt} % Customize the height of the header

\setlength\parindent{0pt} % Removes all indentation from paragraphs - comment this line for an assignment with lots of text

% ----------------------------------------------------------------------------------------
%	TITLE SECTION
% ----------------------------------------------------------------------------------------

\newcommand{\horrule}[1]{\rule{\linewidth}{#1}} % Create horizontal rule command with 1 argument of height

\title{
  \normalfont \normalsize
  \textsc{The University of British Columbia} \\ [25pt] % Your university, school and/or department name(s)
  \horrule{0.5pt} \\[0.4cm] % Thin top horizontal rule
  \huge BMEG 101 - Cellular and Molecular Bioengineering Project Assignment % The assignment title
  \horrule{2pt} \\[0.5cm] % Thick bottom horizontal rule
}

\author{Alex Koen \& Chloe Bolongaro} % Your name

\date{\normalsize \today} % Today's date or a custom date

\begin{document}

\maketitle % Print the title

\onehalfspacing

\begin{enumerate}
    \item 
    \item 
    \item \textit{Calculate the decay length, $\lambda$ of the BMP4 concentration gradient}
    
    \begin{align*}
        \lambda &= \sqrt{\frac{D}{k}} \\
        &= \sqrt{\frac{\SI{1.2}{\micro\m^2\per\s}}{\SI{5.4e-5}{\per\s}}} \\
        &= \SI{149}{\micro\m} \\
    \end{align*}
    \item Dynamics of BMP4:
    
    \begin{align*}
        L &= L_0 e^{\frac{-x}{\lambda}} \\
        \frac{\partial L}{\partial x} &= \frac{-L_0}{\lambda}e^{\frac{-x}{\lambda}} \\
        \frac{\partial ^2L}{\partial x^2} &= \frac{L_0}{\lambda^2}e^{\frac{-x}{\lambda}}
    \end{align*}
    
    Given $\frac{\partial L }{\partial t} = D\frac{\partial^2 L}{\partial x^2}-kL+\nu,$
    
    \begin{align*}
        \frac{\partial L}{\partial t} &= \frac{DL_0}{\lambda^2}e^{\frac{-x}{\lambda}}-kL_0e^{\frac{-x}{\lambda}} + \nu \\
        &= L_0e^{\frac{-x}{\lambda}}\left(\frac{D}{\lambda^2} -k \right) + \nu
    \end{align*}
    
    \textbf{Dynamics of BMPR2:}

    \begin{align*}
    \frac{dR}{dt} &= \alpha L-k_R R \\
      \intertext{and given}
      L &= L_0 e^{\frac{-x}{\lambda}} \\
    \frac{dR}{dt} &= \alpha L_0 e^{\frac{-x}{\lambda}} L-k_R R \\
    \end{align*}

  \item \textit{Find solutions for ligand and receptor equations in steady state.}

      $L_0=\SI{2}{\nano\g\per\micro\m}$

      $\alpha = \SI{5e-3}{\nano\g\per\s\per\micro\m}$

      $k_R= \SI{10e-3}{\per\s}$


      \textbf{Solution for ligand equation in steady-state:}

      In steady state, $\frac{\partial L}{\partial t}$ and $\frac{\partial^2 L}{\partial x^2} = 0$  

      \begin{align*}
        \frac{\partial L }{\partial t} &= D\frac{\partial^2 L}{\partial x^2}-kL+\nu \\
        0 &=-kL + \nu \\
        \nu &= kL \\
        &= k L_0e^{\frac{-x}{\lambda}} \\
          \intertext{To determine the source required at the end of the tube:}
        \nu &= \SI{5.4e-5}{\per\s} \cdot\SI{2}{\nano\g\per\micro\m} \\
              &= \SI{1.08e-4}{\nano\g\per\s\per\micro\m}
      \end{align*}

      \textbf{Solution for receptor equation in steady state:}

      \begin{align*}
        \frac{dR}{dt} &= \alpha L_0 e^{\frac{-x}{\lambda}} L-k_R R \\
        0 &= \alpha L_0 e^{\frac{-x}{\lambda}} L-k_R R \\
        R &= \frac{\alpha}{k_R} L_0e^{\frac{-x}{\lambda}} \\
        \intertext{at the end of the tube ($x=0$):}
        R &= \frac{\alpha}{k_R}L_0 \\
                      &= \frac{\SI{5e-3}{\nano\gram\per\s\per\micro\m}}{\SI{10e-3}{\per\s}} \SI{2}{\nano\gram\per\micro\m} \\
      \end{align*}
        
    \item \textit{Considering the association constant $K_a$, calculate the concentration of the ligand-receptor complex in steady state at the distance from the source $x=\lambda$:}

      \begin{align*}
        K_a &= \frac{K_{on}}{K_{off}} \\
        [RL] &= K_a[R][L] 
      \end{align*}
    \item 
    \item 
    \item 
\end{enumerate}


\end{document}

\implies
%%% Local Variables:
%%% mode: latex
%%% TeX-engine: xetex
%%% End:
